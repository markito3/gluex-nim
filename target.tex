\section[Target (TCR/C. Keith)]{Target \label{sec:target} (TCR/Alexander D.)}
The Hall D cryotarget is a small kapton target cell containing liquid
hydrogen (LH$_2$) or liquid deuterium (LD$_2$) at a temperature and pressure
of about 18~$^\circ$K and 18~psia. The cell, which is closely modeled on those
utilized in Hall B for more than a decade, is a horizontal, tapered
cylinder about 400~mm long with a mean diameter of 20~mm.  Liquid
hydrogen is introduced into the cell using a pair of 1.5~m long
stainless steel tubes (fill and return) connected to a small, copper
vessel where hydrogen gas is condensed.  This condenser is cooled by a
pulse tube refrigerator%
\footnote{Cryomech Model PT415.}
(PTR) with a base temperature of 3~$^\circ$K and
cooling power of about 20~W at 20~$^\circ$K.  
A 100~W temperature controller%
\footnote{Lake Shore Model 336.}
regulates the condenser's temperature.

The entire target assembly is contained in an ``L''-shaped,
stainless steel and aluminum vacuum chamber with a Rohacell extension
surrounding the target cell.  The vacuum chamber, along with the
hydrogen storage tanks, gas handling system, and control electronics,
is mounted on a custom-built beamline cart for insertion into the Hall
D solenoid.  A Programmable Logic Controller (PLC) monitors and
controls the performance of the target, while hardware interlocks on
the target temperature and pressure and on the chamber vacuum ensure
the system's safety and integrity.

A simplified schematic drawing of the target system is shown in 
Fig.~\ref{fig:tgt:scheme}, and a scale drawing of the target, 
with the insertion cart, is shown in Fig.~\ref{fig:tgt:pict}.


% ======================================================================================

\begin{figure}[h]
\begin{center}
    %\includegraphics[angle=0,width=0.90\linewidth]{figs/cryo_target_schematics_1}
\end{center}
\caption{
Highly simplified schematic of the target system.  Not to scale.  The
Rohacell vacuum chamber extension is indicated in green.  The photon
beam travels from left to right in this figure.
\label{fig:tgt:scheme}}
\end{figure}

% ======================================================================================

\begin{figure}[h]
\begin{center}
      %\includegraphics[angle=0,width=0.90\linewidth]{figs/cryo_target_pict}
\end{center}
\caption{
Scale drawing of the target assembly.  The Rohacell extension is
removed and the Kapton target cell is visible on the far left.  The
hydrogen storage tanks are yellow, and the pulse tube refrigerator is
blue.
\label{fig:tgt:pict}}
\end{figure}

% ======================================================================================

\subsection[Target Cell]{Target Cell \label{sec:tgt:cell}}

A detailed drawing of the target cell is shown in
Fig.~\ref{fig:tgt:cell}.  The cell is 5~mil (0.13~mm), alum-inized
polyimide (Kapton) foil wrapped into a tapered cylinder and glued
along one edge.  An extruded Kapton exit window is glued to the
downstream end, and the upstream end is glued into an aluminum base.
A polyimide-amide (torlon) tube is glued into the aluminum base and
acts as a re-entrant window frame for a Kapton beam-entrance window.
Stainless steel tubes for liquid hydrogen (fill and return) are also
glued into the base.  The epoxy%
\footnote{3M Scotch-Weld DP-190 (gray).}
used for all
glued joints has a long and successful track record for cryogenic use
at Jefferson Lab.  Two calibrated Cernox4 thermometers will be
inserted into the cell to determine the temperature with better than
0.5\% accuracy.

Prototypes for these cells have been constructed and pressurized to
failure at room temperature.  The burst pressure is typically around
200~psia. Based on the results of pull tests performed at 77~$^\circ$K, the
burst pressure of the cell at 18~$^\circ$K will be significantly higher than
at room temperature.  However we do take credit for this in our
pressure relief analysis.  The operational pressure of the cell is
about 35~psia at room temperature and about 18~psia when filled with
LH$_2$. Redundant pressure relief valves, along with properly sized
system plumbing, ensure that the cell pressure never exceeds 45~psia.
All relief paths, including one on the vacuum chamber as well as all
pump exhausts are directed to a dedicated hydrogen vent line installed
in Hall D.  This line terminates outside the hall and is continuously
purged with an inert gas (nitrogen) to ensure that a flammable
hydrogen/oxygen mixture does not exist inside the line.

% ======================================================================================

\begin{figure}[h]
\begin{center}
      %\includegraphics[angle=0,width=0.98\linewidth]{figs/cryo_target_cell}
\end{center}
\caption{
Hall D liquid hydrogen target cell.
\label{fig:tgt:cell}}
\end{figure}

% ======================================================================================
 
\subsection[Condensation of Hydrogen Gas]{Condensation of Hydrogen Gas 
\label{sec:tgt:condens}}

Room temperature hydrogen gas is introduced to the target from a pair
of 55-gallon (208~$\ell$) storage tanks located on the insertion cart
and charged with an initial H$_2$ pressure of approximately 35~psia (2.3
atm).  The gas is first cooled to approximately 80~$^\circ$K by a copper heat
exchanger in thermal contact with the first stage of the pulse tube
refrigerator (PTR).  The first stage of the PTR is also used to cool a
copper heat shield that surrounds the condenser and the fill and
return tubes to the target cell.  The shield does not extend to the
cell itself, which is instead wrapped with ten layers of
superinsulation.

