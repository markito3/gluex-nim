%=======================+=========================
%================  Performance  ================
%=================================================

\section[Detector performance (Sean)]{Detector performance \label{sec:performance}}

The performance of the GlueX detector has been studied in data and simulation using several photoproduction reactions.  The results of these studies are summarized in this section.

\subsection{Charged particles \label{sec:perfcharged}}

\subsubsection{Efficiency \label{sec:perfchargedeff}}

The reconstruction efficiency for different hadron species has been studied using the method described in Sec.~\red{sec:trackeff}.   Charged pion reconstruction efficiency was studied with a sample of $\gamma p \to p \omega$, $\omega \to \pi^+\pi^-\pi^0$ events.  Charged kaon reconstruction efficiency was studied with samples of $\gamma p \to \phi p$, $\phi \to K^+K^-$ and $\gamma p \to \Lambda(1520) K^+$, $\Lambda(1520) \to p K^-$ events.  Proton reconstruction efficiency was studied with a sample of $\gamma p \to K^+ \Sigma^0$, $\Sigma^0 \to \gamma \Lambda^0$, $\Lambda^0 \to p \pi^-$ events.  
The results are illustrated in Fig.~X. The efficiencies determined by data and simulation generally agree to within a few percent.

\subsubsection{Resolution \label{sec:perfchargedresol}}

The invariant mass resolution for resonances decaying into charged particles has been studied using several different channels:  $K_S\to\pi^+\pi^-$ from $\gamma p \to K_S K+ \pi^- p$, $\Xi^- \to \pi^- \Lambda^0$, $\Lambda^0 \to p \pi^-$ from $\gamma p \to K^+ K^+ \Xi^-$, and $J/\psi \to e^+ e^-$ from $\gamma p \to J/\psi p$.  The results are illustrated in Fig.~X.  In each case, the invariant mass resolution was in the range $10-20?$~MeV, after applying a kinematic fit to the full reaction.

\subsection{Neutral particles \label{sec:perfneutral}}

\subsubsection{Resolution \label{sec:perfneutralresol}}

The invariant mass resolution of the decay $\eta \to \gamma \gamma$ has been found to primarily depend on the energy resolution of the calorimeters at GlueX.  Therefore, in Fig.~X we show the resolution for these reconstructed decays for three classes of events: where both photons from the  $\eta \to \gamma \gamma$ are reconstructed in the BCAL, were both are reconstructed in the FCAL, and where one photon is reconstructed in the BCAL and one in the FCAL.

\subsubsection{Efficiency \label{sec:perfneutraleff}}

Photon reconstruction efficiency has been studied using different methods for the FCAL and BCAL.  In the FCAL, absolute photon reconstruction efficiencies have been determined using the ``tag-and-probe'' method with a sample of photons from the reaction $\gamma p \to \omega p$, $\omega \to \pi^+\pi^-\pi^0$, $\pi^0 \to \gamma (\gamma)$, where the final photon is allowed but not required to be reconstructed.  The yields with and without the reconstructed photon are determined using two methods.  In the first method, the $\omega$ meson yield is determined by a fit to the $M(\pi^+\pi^-\pi^0)$ spectrum before and after a cut on the $M(\gamma\gamma)$ requiring a fully-reconstructed $\pi^0$.  In the second method, if the desired photon is found to be reconstructed, the  $\omega$ meson yield is determined as above.  If the photon is not reconstructed, the $\omega$ yield is determined by a fit to the distribution of the missing mass off of the proton.
Both methods yield consistent results, with a reconstruction efficiency generally above 90\%, and agree with the efficiencies determined from simulation within less than 5\%.

In the BCAL, a relative photon efficiency determination has been performed using $\pi^0\to\gamma\gamma$ decays.  

[say something about fiducial cuts?]

\subsection{Kinematic fitting \label{sec:perffitting}}



\subsection{Particle identification \label{sec:perfpid}}

\subsection{Systematic uncertainties \label{sec:systematics}}
