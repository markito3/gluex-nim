%=======================+=========================
%================  Performance  ================
%=================================================

\section[Detector performance (Sean)]{Detector performance \label{sec:performance}}

The performance of the GlueX detector has been studied in data and simulation using several photoproduction reactions.  The results of these studies are summarized in this section.

\subsection{Charged particles \label{sec:perfcharged}}

\subsubsection{Efficiency \label{sec:perfchargedeff}}

The reconstruction efficiency for different hadron species has been studied using the method described in Sec.~\ref{sec:trackeff}.   Charged pion reconstruction efficiency was studied with a sample of $\gamma p \to p \omega$, $\omega \to \pi^+\pi^-\pi^0$ events.  Charged kaon reconstruction efficiency was studied with samples of $\gamma p \to \phi p$, $\phi \to K^+K^-$ and $\gamma p \to \Lambda(1520) K^+$, $\Lambda(1520) \to p K^-$ events.  Proton reconstruction efficiency was studied with a sample of $\gamma p \to K^+ \Sigma^0$, $\Sigma^0 \to \gamma \Lambda^0$, $\Lambda^0 \to p \pi^-$ events.  
The results are illustrated in Fig.~X. The efficiencies determined by data and simulation generally agree to within a few percent.

\subsubsection{Resolution \label{sec:perfchargedresol}}

The invariant mass resolution for resonances decaying into charged particles has been studied using several different channels:  $K_S\to\pi^+\pi^-$ from $\gamma p \to K_S K+ \pi^- p$, $\Xi^- \to \pi^- \Lambda^0$, $\Lambda^0 \to p \pi^-$ from $\gamma p \to K^+ K^+ \Xi^-$, and $J/\psi \to e^+ e^-$ from $\gamma p \to J/\psi p$.  The results are illustrated in Fig.~X.  In each case, the invariant mass resolution was in the range $10-20?$~MeV, after applying a kinematic fit to the full reaction.

\subsection{Neutral particles \label{sec:perfneutral}}

\subsubsection{Resolution \label{sec:perfneutralresol}}

The invariant mass resolution of the decay $\eta \to \gamma \gamma$ has been found to primarily depend on the energy resolution of the calorimeters at GlueX.  Therefore, in Fig.~X we show the resolution for these reconstructed decays for three classes of events: where both photons from the  $\eta \to \gamma \gamma$ are reconstructed in the BCAL, were both are reconstructed in the FCAL, and where one photon is reconstructed in the BCAL and one in the FCAL.

\subsubsection{Efficiency \label{sec:perfneutraleff}}

Photon reconstruction efficiency has been studied using different methods for the FCAL and BCAL.  In the FCAL, absolute photon reconstruction efficiencies have been determined using the ``tag-and-probe'' method with a sample of photons from the reaction $\gamma p \to \omega p$, $\omega \to \pi^+\pi^-\pi^0$, $\pi^0 \to \gamma (\gamma)$, where the final photon is allowed but not required to be reconstructed.  The yields with and without the reconstructed photon are determined using two methods.  In the first method, the $\omega$ meson yield is determined by a fit to the $M(\pi^+\pi^-\pi^0)$ spectrum before and after a cut on the $M(\gamma\gamma)$ requiring a fully-reconstructed $\pi^0$.  In the second method, if the desired photon is found to be reconstructed, the  $\omega$ meson yield is determined as above.  If the photon is not reconstructed, the $\omega$ yield is determined by a fit to the distribution of the missing mass off of the proton.
Both methods yield consistent results, with a reconstruction efficiency generally above 90\%, and agree with the efficiencies determined from simulation within less than 5\%.

In the BCAL, a relative photon efficiency determination has been performed using $\pi^0\to\gamma\gamma$ decays.  We look at a sample of fully reconstructed $\gamma p \to  \pi^+\pi^-\pi^0 p$ events, and use the fact that the $\pi^0\to\gamma\gamma$ decay is isotropic in the center of mass frame.  Therefore, any anisotropy reflects an inefficiency in the detector. Results from this analysis are illustrated in Fig.~X. Generally, this relative efficiency is above 90\%, and agrees with that determined from simulation within 5\%.

[say something about fiducial cuts?]

\subsection{Kinematic fitting \label{sec:perffitting}}

Kinematic fitting is a powerful tool to improve the resolution of measured particles and to distinguish between different reactions.  In GlueX, this method leverages the fact that the initial state is very well known, with the target proton essentially at rest, and the incident photon energy is measured with very high precision ($<0.1\%$).  The most common kinematic fits that are performed are those that impose energy-momentum conservation between the initial and final state particles.  Additional optional constraints in these fits are for the four-momenta of the daughters of an intermediate particle to add up to a fixed invariant mass, and for all the particles to come from a common vertex (or multiple vertices, in the case of reactions containing long-lived, decaying particles).

To illustrate the performance of the kinematic fit, we show the improvement in the invariant mass resolution in the reaction $\gamma p \to \eta p$, $\eta \to \pi^+\pi^-\pi^0$ in Fig.~X, and the pull distributions from the fit are illustrated in Fig.~X2.

\subsection{Particle identification \label{sec:perfpid}}

Particle identification in GlueX uses information from both energy loss in different detector systems and time-of-flight measurements.  This information can be used for identification in several ways.  The simplest method is to apply selections directly on the relevant PID variables.  To include detector resolution information, one can create a $\chi^2$ variable comparing a measured value to the expected value for a particular hypothesis, that is
\begin{equation}
    \chi^2(p) = \left(  \frac{ X(\mathrm{measured}) - X(\mathrm{expected})_p}{\sigma_X} \right)
\end{equation}
where $X$ is the given PID variable, $p$ is the particle hypothesis, and $\sigma_X$ is the resolution of this variable.  Finally, multiple PID variables can be combined into one probability, or other figure-of-merit (FOM).

At sufficiently large $\theta$, the energy loss ($dE/dx$) for charged particles in the central drift chamber can be used.   Fig.~X illustrates these distributions for positive and negative charged particles, along with the standard selection used to separate pions and protons.

The primary means of particle identification is through time of flight measurements, and information from several sources are combined to make the most accurate determination.  The RF reference signal from the accelerator is used to time when each photon bunch enters the target.  The reconstructed final state particles are used to determine which photon bunch most likely generated the detected reaction, with the primary determination being the signals from the Start Counter associated with the charged particle tracks.  This photon bunch determination has a resolution of around 15~ps?. Each charged particle is associated with additional timing information based on the hit in the highest resolution detector that it is associated with, for example the BCAL or TOF.  The time of flight to this measured hit ($t_\mathrm{meas.}$) from the time of the photon bunch that generated the event ($t_\mathrm{RF})$ can be used to distinguish between differently massed particles.  Two common variables that are used are the velocity ($\beta$) determined using the measured time-of-flight and the momentum of the particle, and $\Delta t_\mathrm{RF}$, the difference between the measured and RF times described above after they both have been extrapolated back to the center of the target, assuming some particle hypothesis.
Some illustrations are given in the associated figures.

Electrons are primarily identified using the ratio of their energy loss in the electromagnetic calorimeters ($E$) to the momentum reconstructed in the drift chambers ($p$).  This $E/p$ ratio should be approximately 1 for electrons and generally less for hadrons.  This is illustrated in some sample of particles for Fig.~X.

Particle identification and mis-identification rates are summarized in Fig.~X.

\subsection{Systematic uncertainties \label{sec:systematics}}
