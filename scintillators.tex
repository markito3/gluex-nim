\section[Scintillation detectors (Mark I./Beni)]{Scintillation detectors \label{sec:scintillators}}
\subsection{Start counter \label{sec:st}}

The Start Counter (ST) detector, shown in Fig.~\ref{fig:st-overview-drawing}
surrounds the target
region and covers about 90\% of the solid angle for particles
originating from the center of the target. It is designed to operate
at tagged photon beam intensities of up to $10^8$ photons per second
in the coherent peak. It has a high degree of segmentation to limit
the the per-paddle rates. The time resolution is less than 350 ps
RMS. The Start Counter provides a timing signal that is releatively
independent of particle type and trajectory (because of its proximity
to the target), can provide particle identification via $dE/dx$, and
can be used in the Level 1 trigger.

\begin{figure}[!htb]
\centering
\includegraphics[width=1.0\columnwidth]{figures/start_counter_all.pdf}
\caption{The \gx{} Start Counter mounted to the liquid $\mathrm{H_2}$
  target assembly.  The beam goes from left to right down the central
  axis.\label{fig:st-overview-drawing}}
\end{figure}

\subsubsection{Detector Description}

The ST consist of 30 scintillator paddles arranged in a cylinder for
most of its length with a ``nose'' section that bends towards the beam
line at the downstream end. EJ-200 scintillator from Eljen
Technology\footnote{Eljen Technology,
  https://eljentechnology.com/products/plastic-scintillators} was
selected. EJ-200 has a decay time of 2.1~ns and an attenuation length
of 380~cm. Silicon photomultiplier (SiPM) detectors were used as light
sensors. These are not affected by the magnetic field produced by the
GlueX solenoid, an important feature since the start counter is located
near the geometric center of the field region. The SiPMs were placed
as close as possible to the upstream end of each scintillator element
to maximize light collection.

Each scintillator paddle started from stock 3~mm thick and 600 mm in
length. The paddles were bent at Eljen to create the nose section, and
then machined at McNeal Enterprises Inc.\footnote{McNeal
  Enterprises~Inc., http://www.mcnealplasticmachining.com} to their
final shape, including edges beveled at $6^\circ$ to minimize loss of
acceptance.

The scintillabor paddles are supported by a Rohacell closed-cell foam
structure. The Rohacell is 11~mm thick and is rigidly attached to an
aluminum support hub at its upstream end. The downstream support
extends partially into the nose section. The cylindrical length of the
Rohacell is further reinforced with three layers of carbon fiber, each
layer 650~$\mu$m thick. The assembly is made light-tight with a Tedlar
wrapping. The Tedlar is attached to a plastic collar at the upstream
end.

\subsubsection{Read-Out Electronics}

Each scintillator bar is read out with an array of four
magnetic-field-insensitive Hamamatsu S109031-050P multi-pixel photon
counters (the SiPMs). Each SiPM consists of 3,600 $50 \times 50$
$\mu$m$^2$ avalanche photo-diode pixel counters operating in Geiger
mode. The signals from all pixels are summed. The scintillator is
optically coupled to the SiPMs through a 250 $\mu$m air gap.

The read-out electronics are deployed on three separate board
designs. The first board houses the SiPMs themselves as well as a
thermocouple for temperature monitoring. Each of these boards services
three scintillator paddles (twelve SiPMs per board). There are ten
boards altogether.

The second board provides a pre-amplifier for each channel whose
output is split. One side of the signal goes to a buffer and then to a
Flash ADC (FADC), the other goes to a 5$\times$~amplifier and is sent
from there to a discriminator and TDC. Low-voltage power and the bias
voltage for the SiPMs are routed. All four SiPMs for one scintillator
receive a single bias voltage, but that voltage is set separately for
each set of four.

The third board provides an interface for distribution of the
low-voltage power for the electronics from the external power supply,
as well as for the bias voltages for the SiPMs. It also serves as a
patch panel for the FADC, discriminator, and thermocouple signal
output cables.

\subsubsection{Calibration}

Calibrations are performed to correct measurements for the effects of
time-walk, light propagation time, and light attenuation

\begin{description}
\item{\bf Time-Walk Correction}. Since the detector signal is sent to
  both a Flash ADC and a TDC, the time from the the FADC, which is
  largely independent of pulse amplitude, is used to measure the time
  walk seen by the TDC as a function of pulse amplitude as measured by
  the FADC. The resulting curve is fit to an empirical function to
  apply the correction and the procedure is done on a
  channel-by-channel basis.
\item{\bf Light Propagation Time Correction}. The propagation time is
  measured as a function of the hit position in a paddle as determined
  by well-reconstructed charged particle tracks. The propagation
  velocity is measured in three regions of the counter (``straight,''
  ``bend,'' and ``nose'') and is not assumed to be a single value for
  all hits. The parameters for the correction are obtained
  independently for each paddle.
\item{\bf Light Attenuation Correction}. Light attenuation is measured
  at several positions along the counter using charged particle tracks
  again. The energy per unit path length in the paddle as a function
  of distance from the SiPMs is fit to a modified exponential, with
  different parameters allowed for the straight section versus the
  nose with continuity enforced at the section boundary.
\end{description}

\subsubsection{Performance}

\smallskip

{\it Timing Resolution}

\smallskip

Timing performance can be measured by comparing the event time in the
target as measured by the start counter and the time derived from a
signal from the CEBAF accelerator which is synched to the RF time
structure of the machine. The start counter time must be corrected for
the flight path of the charged particle emerging from the event and
all instrumental corrections mentioned in the previous section are
applied. Fig.~\ref{fig:st-time-resolution} shows the distribution of
this time difference. The time from the RF does not contribute
significantly to the width of the
distribution. Table~\ref{table:st-time-resolution} gives the measured
time resolution for the various sections as well as for all sections,
with all paddles combined. Also shown is the fraction of tracks kept
by a $\pm 1.0$~ns cut around the central value.

\begin{figure}[!htb]
  \centering
  \includegraphics[width=1.0\linewidth]{figures/st_tr_fit.pdf}
  \caption{Time resolution for one paddle with its full width half
    maximum value indicated in ns.  The x-axis is the time difference
    between $T^{\rm ST}_{\rm vertex}$ and $T^{\rm BB}_{\rm vertex}$.
    The vertical lines indicate the cuts used to identify a 500~MHz beam bunch.}
                \label{fig:st-time-resolution}
\end{figure}  

\begin{table}[htbp]
  \centering
  \begin{tabular}{@{} l *4c @{}}
    \hline
    \multicolumn{1}{c}{\textbf{Section}}    & \textbf{All}  & \textbf{Straight}  & \textbf{Bend}  & \textbf{Nose}  \\ 
    \hline
    $\mathbf{FWHM}$ & 550~ps & 690~ps & 700~ps & 450~ps \\ 
    \textbf{Fraction} & 93\% & 92\% & 91\% & 94\% \\\hline
  \end{tabular}
  \caption{Average time resolutions (FWHM) and event fractions within a $\pm$ 1~ns window for all 30 ST sectors by independent geometrical regions.}
  \label{table:st-time-resolution}
\end{table}  

\smallskip

{\it Particle Identification via dE/dx}

\smallskip

Energy loss per unit path length ($dE/dx$) measured in the start
counter can be used for particle
identification. Fig.~\ref{fig:ST_dEdx_vs_p} shows $dE/dx$ vs. momentum
for a charged particles matched to the Start Counter. Protons can be
separated from pions up to 0.9 GeV/$c$ in momentum.

\begin{figure*}[!htb]
  \centering
  \includegraphics[width=\textwidth]{figures/st_dedx_vs_p.pdf}
  \caption{$dE/dx$ vs.\ $p$ for the Start Counter.  The curved band
    corresponds to protons while the horizontal band corresponds to
    electrons, pions, and kaons. Pion/proton separation is achievable
    for tracks with $p < 0.9$~GeV/$c$.}\label{fig:ST_dEdx_vs_p}
\end{figure*}

\subsection[Time-of-flight counters (Beni)]{Time-of-flight counters \label{sec:tof}}
The Time-of-flight (TOF) detector is a wall of scintillators located about 5.5~m downstream from the target and covers 
an angular region from 0.6$^{\circ}$ to 13$^{\circ}$ in polar angle. The detector has two planes of
scintillator paddles stacked in the horizontal and vertical direction, respectively. Most paddles are 252~cm long and 2.54~cm
thick with a width of 6~cm. 
The scintillator material is EJ-200 from Eljen technology.
To accommodate the photon beam to pass through the central region,
an aperture of 12$\times$12\,cm$^2$ is kept
free of any detector material giving rise to four short paddle detectors with a length of 120~cm around the beam hole
in each detector plane. These paddles also have a width of 6~cm with a thickness of 2.54~cm. In order to keep the
count rate of the paddles well below 2~MHz the two inner most full-length paddles closest to the beam hole have a reduced width of 3~cm.
Light guides from UV transmitting plastic provide the coupling space between the scintillator and the PMT and allow the 
magnetic shielding to protect the photo cathode by extending about 5~cm past the PMT entrance window. All paddles are wrapped
with a layer of a highly reflective material DF2000MA from 3M followed by a layer of strong black Tedlar film for light tightness. 
The main purpose of the detector is to provide fast timing for charged particles passing through the detector thereby providing information for particle identification and to the determine the event RF beam bunch of the photon that initiated the event.

\subsection{Electronics \label{sec:scelectronics}}
The scintillator paddles are read out using 
photomultiplier tubes (PMT) from Hamamatsu~\footnote{Hamamatsu Photonics, https://www.hamamatsu.com/us/en/index.html.}. Full-length paddles
have a PMT at both ends, while the short paddles have a single PMT
at the outer edge of the detector. These tubes of type H10534 have 10-stages and are complete assemblies with high voltage base, casing and $\mu$-metal shielding. Due to the significant stray field from the spectrometer solenoid magnet, additional external
shielding based on soft iron is necessary to protect the PMTs from the magnetic field.
The high voltage (HV) to the PMTs is provided by CAEN HV modules of type A1535SN initially controlled by a CAEN SY1527 main frame and
later upgraded to a SY4527.
The PMT output is connected to a splitter by a 50' long cables (WHAT TYPES?). The signal is split by
a passive splitter into two equal-amplitude signals. One signals is directly connected to a flash ADC250
analog to digital converter (fADC)~\cite{Dong:2007}, while the second signal passes first through a leading edge discriminator (LED) and then used as an input to a high resolution TDC VX1290A from CAEN~\footnote{CAEN "https://www.caen.it/"}. The digitizer electronics (fADC250 and TDC) are mounted in VXS crates as described in Section~\ref{sec:trigdaq}.
The threshold of the leading edge discriminator is controlled for each channel separately and has an intrinsic
dead time of about 25~ns.
The sparcification threshold for the fADC250 is set to 105???? counts, with the nominal pedestal set at 100. The data from the fADC250 is provided by the FPGA algorithm and consists
of two words per channel with information about pedestal, signal amplitude, signal integral and timing.
The VX1290A TDC is a multi-hit high-resolution TDC with a buffer of 
up to 8 words per channel. The intrinsic dead time of a TDC channel is ???~ns. One TDC count is approximately 25~ps.
Since these TDCs provide the best time measurements in the GlueX detector the timing of the accelerator RF signal is also
digitized using this electronics.

\subsection{Calibration and monitoring \label{sec:sccalib}}
A detailed description of the TOF detector can be found in Ref.\,\cite{GlueXTOFNIM}. Since the TOF detector consists of two
planes of narrow paddles oriented orthogonal to each other it is possible to calibrate the full detector independent
of any other external detector information. The overlap region of two full-length paddles from the two planes define
a 6$\times$6~cm$^2$ area for most paddles with a few 3$\times$3~cm$^2$ areas close to the beam hole. The separation between
the two detector planes is minimal as they are mounted on top of each other and as such are only separated by their wrapping
material. While the time-difference TD between the two ends of a paddle is related to the hit position along the paddle
the mean-time MT is related to the flight time of a particle from the vertex to the paddle. Therefor the MT for two overlapping
paddles must be the same when they are hit by the same particle passing through both of them while the hit position in the horizontal (x) and vertical (y) dimensions are defined by TD of the two paddles. This relationship results in an internally consistent calibration of all paddles with respect to every other paddle.

Prior to finding timing offsets, all times must be corrected for time-walk because of the use of LED discriminators, which
introduce a time shift that depends on the signal amplitude. The relation between time at threshold and signal amplitude is parameterized and used to correct for time slewing.

After all full-length paddles have been calibrated, they can be used themselves as references to
calibrate the remaining 8 short paddles that only have single-ended readout.  Again we use the fact that any overlap region of two paddles from different
planes has the same particle flight time from the vertex. This coincidence produces peaks in the time difference distributions that can be used to determine the timing offsets of these single-ended readout paddles. 

To test the calibration, we take tracks that are incident on a paddle in one plane and compute the time difference between the MT of that paddle and the MT of every other full-length paddle in the other plane. The resulting distribution of these differences is shown in Fig.~\ref{fig:mt_diff}. Assuming that all paddles have the same timing resolution, we can compute the
average time resolution to be $\sigma$ = 96~ps$=\frac{136}{\sqrt{2}}$~ps, assuming a Gaussian distribution.
\begin{figure}[tbp]
\begin{center}
\includegraphics[width=0.8\textwidth]{figures/mt_diff_fullTOF.pdf}
\caption{\label{fig:mt_diff} Mean time difference between one long paddle of one plane with all other long paddles
of the other plane. (Color online)}
\end{center}
\end{figure}
\subsection{Performance \label{sec:scperformance}}
To investigate the performance of the TOF detector for its PID capability it is important to include the relative number of
particle types within the event sample. We select events that have at least three fully reconstructed positively charged tracks with at least one of these tracks intersecting the TOF detector. We expect more pions than protons, and more protons than kaons. Looking at the distribution of velocity ($\beta$) of these tracks as a function of momentum it is easy to identify the bands from protons, kaons and pions (see Fig.~\ref{fig:betavsp}). 

The distributions of $\beta$ at two specific track momenta, 2~GeV/c and 4~GeV/c (see Fig.~\ref{fig:betaproj}), are very illustrative of the PID capability of the TOF detector. At 2~GeV/c particle momentum the TOF detector provides about a 4$\sigma$ separation between
the pion/positron peak and the kaon peak. This is sufficient to identify tracks with a $\beta$ of 0.97 or lower as kaons with a very
high certainty. However, at a $\beta$ of 0.98 the probability of the track begin a kaon is less than 50\% mainly due to the fact
that the abundance of pions is close to an order of magnitude larger than kaons. The protons, on the other hand, are very well
separated from the other particle types and can be identified as such with high confidence over the full range in $\beta$.
At a track momentum of 4~GeV/c, this has changed and represents the limit at which the TOF can identify protons with high confidence. Again the separation between the large peak containing pions, kaons and positrons is separated from the proton
peak by about 4$\sigma$, while the relative abundance in this case is about a factor of 4. As a consequence, a 4~GeV/c momentum
track with a $\beta$ of 0.975 is most likely a proton with a small probability of being a pion. At a $\beta$ of 0.98 such
a track has a similar probability for being a proton or a pion.
\begin{figure}[tbp]
\begin{center}
\includegraphics[width=0.8\textwidth]{figures/beta_vs_p_positivetracks.pdf}
\caption{\label{fig:betavsp}$\beta$ of positive charged track vs track momentum. The color coding of the third dimension
is in logarithmic scale.(Color online)}
\end{center}
\end{figure}

\begin{figure}[tbp]
\begin{center}
\includegraphics[width=0.45\textwidth]{figures/TOF_postracks_2000mev.pdf}
\includegraphics[width=0.45\textwidth]{figures/TOF_postracks_4000mev.pdf}
\caption{\label{fig:betaproj}$\beta$ of positive charged track with 2~GeV/c momentum (left) and with 4~GeV/c (right).}
\end{center}
\end{figure}

\subsection{Summary \label{sec:scsummary}}
The TOF detector in the forward region of the GlueX spectrometer provides high resolution timing information that contributes
to the identification of the RF beam bucket of the beam photon that initiated the event. In combination with the charged
track reconstruction, TOF hits can be matched to such tracks to determine $\beta$ and give access to the particle
mass of the track. Protons can be identified with reasonable confidence up to momenta of 4~GeV/c while kaons can be
identified up to momenta of 2~GeV/c.
